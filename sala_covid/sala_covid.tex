%

%Note: LaTeX Beamer taken from the website: https://www.sharelatex.com/templates/presentations/conference-presentation
%Note: to find out the RGB code for dolphin: https://tex.stackexchange.com/questions/66465/how-to-get-actual-values-of-colour-theme-colours-in-beamer
%Note: to convert eps files to pdf: https://docupub.com/pdfconvert/

\documentclass[xcolor=table]{beamer}
\usepackage{appendixnumberbeamer}

\usepackage{subcaption}

% Change the margin
%\setbeamersize{text margin left = 2pt, text margin right = 2pt}

\setbeamertemplate{footline}[frame number]
\setbeamertemplate{headline}{}
% There are many different themes available for Beamer. A comprehensive
% list with examples is given here:
% http://deic.uab.es/~iblanes/beamer_gallery/index_by_theme.html
% You can uncomment the themes below if you would like to use a different
% one:
%\usetheme{AnnArbor}
%\usetheme{Antibes}
%\usetheme{Bergen}
%\usetheme{Berkeley}
%\usetheme{Berlin}
%\usetheme{Boadilla}
%\usetheme{boxes}
%\usetheme{CambridgeUS}
%\usetheme{Copenhagen}
%\usetheme{Darmstadt}
\usetheme{default}
%\usetheme{Frankfurt}
%\usetheme{Goettingen}
%\usetheme{Hannover}
%\usetheme{Ilmenau}
%\usetheme{JuanLesPins}
%\usetheme{Luebeck}
%\usetheme{Madrid}
%\usetheme{Malmoe}
%\usetheme{Marburg}
%\usetheme{Montpellier}
%\usetheme{PaloAlto}
%\usetheme{Pittsburgh}
%\usetheme{Rochester}
%\usetheme{Singapore}
%\usetheme{Szeged}
%\usetheme{Warsaw}
\makeatletter
\definecolor{beamer@blendedblue}{RGB}{252, 196, 108} % changed this

%\setbeamercolor{normal text}{fg=black,bg=white}
%\setbeamercolor{alerted text}{fg=red}
%\setbeamercolor{example text}{fg=green!50!black}

\setbeamercolor{structure}{fg=beamer@blendedblue}
%\useinnertheme{rounded}

%\renewcommand{\labelitemi}{$\bullet$}
%\renewcommand{\labelitemii}{$\cdot$}
%\renewcommand{\labelitemiii}{$\diamond$}
%\renewcommand{\labelitemiv}{$\ast$}

%colors
\usecolortheme{dolphin}
\usepackage{color}
\definecolor{mycolor1}{RGB}{236, 244, 244}
\definecolor{mycolor2}{RGB}{164, 92, 92}	
\definecolor{mycolor3}{RGB}{252, 196, 108}
\definecolor{mycolor4}{RGB}{76, 60, 52}
\definecolor{mycolor5}{RGB}{164, 248, 208}
\definecolor{mycolor6}{RGB}{116, 112, 112}
\definecolor{mycolor7}{RGB}{116, 116, 52}

\usepackage{fnpct}

\usepackage{subcaption}

\usepackage{newpxtext}


%fonts
\usefonttheme{professionalfonts}

%Other packages
\usepackage{bm}
\setbeamercovered{transparent}
\usepackage[section]{placeins}
\usepackage{epstopdf}
\usepackage[table]{xcolor}
\usepackage{graphicx} %Useful to resize large tables into the frame using \resizebox 
\usepackage{grffile} %Documentation: http://ctan.sharelatex.com/tex-archive/macros/latex/contrib/oberdiek/grffile.pdf
\usepackage{marvosym}
\usepackage[style=british]{csquotes}
\def\signed #1{{\leavevmode\unskip\nobreak\hfil\penalty50\hskip2em
		\hbox{}\nobreak\hfill #1
		\parfillskip=0pt \finalhyphendemerits=0 \endgraf}}

\newsavebox\mybox
\newenvironment{aquote}[1]
{\savebox\mybox{#1}\begin{quote}}
	{\vspace*{1mm}\signed{\usebox\mybox}\end{quote}}

\usepackage{verse}
\newcommand{\attrib}[1]{
	\nopagebreak{\raggedleft\footnotesize #1\par}}

%for checkmarks
\usepackage{tikz}

%For equations
\usepackage{amsmath, amssymb}
\usepackage{bbm}

%Tables
\usepackage{tabularx}
\usepackage{multirow}
\usepackage{multicol} 
\usepackage{booktabs}%\usepackage{booktabs, calc} %This is the package to use to have nice-looking tables. More documentation on the tables in LateX: https://www.tug.org/pracjourn/2007-1/mori/mori.pdf
\usepackage{threeparttable}  

\usepackage{lmodern}
\usepackage{booktabs}
\usepackage{pgfplots}

% Count slides
\setcounter{MaxMatrixCols}{10}

\graphicspath{{../figuras/}}
%\graphicspath{{../tablas/}}

% Commands -
\newcommand{\extractRGB}[1]{\extractcolorspecs{#1}{\model}{\dolphin} \convertcolorspec{\model}{\dolphin}{RGB}\printcol \printcol}
\setbeamertemplate{caption}{\raggedright\insertcaption\par} %to prevent beamer from putting "figure" in front of a caption
\setbeamertemplate{navigation symbols}{}

%Code to create sections with title pages in Beamer slides
\AtBeginSection[]{
	\begin{frame}[plain]
		\vfill
		\centering
		\begin{beamercolorbox}[sep=8pt,center,shadow=true,rounded=true]{title}
			\usebeamerfont{title}\insertsectionhead\par%
		\end{beamercolorbox}
		\vfill
	\end{frame}
}

\title{Sala Situacional COVID-19, Región Cusco\footnote{Análisis con información de la región hasta el 23 de octubre del 2021}}
\author{Fatima Concha \& Johar Cassa} 
\institute{Dirección de Epidemiología e Investigación \\ \textbf{\color{mycolor3}DEIS, GERESA - Cusco}}
\date{Semana Epidemiológica 42, 2021}

\begin{document}
	%------------------------------------------------------------------------------------------------------------------------------------------------------------------------------------------------------------------------------------------
	% TITLE PAGE 
	%------------------------------------------------------------------------------------------------------------------------------------------------------------------------------------------------------------------------------------------
	\setbeamercovered{invisible}
	\begin{frame}[plain]
		\titlepage
	\end{frame}

	%------------------------------------------------------------------------------------------------------------------------------------------------------------------------------------------------------------------------------------------
	% INTRODUCTION 
	%------------------------------------------------------------------------------------------------------------------------------------------------------------------------------------------------------------------------------------------
	
	\setcounter{subsection}{1}
	\begin{frame}[label=indice]
		\frametitle{Índice}
		\vspace{-.5cm}
		\begin{itemize}
			\item Indicadores epidemiológicos.
			\begin{itemize}
				\item Nacional: Variantes, cobertura vacuna, mortalidad, RT, ocupación de camas UCI y no UCI a nivel regional. \hyperlink{epi_nacional}{\beamergotobutton{epi. nacional.}}
				\item Regional: Curva epidémica, incidencia, tasa de positividad, variantes ({\color{mycolor5}Nuevo} \hyperlink{variantes}{\beamergotobutton{variantes}}), defunciones, RT por distrito, exceso de defunciones, letalidad, mortalidad, y mortalidad por grupos de edad e hitos de vacunación.  \hyperlink{epi_cusco}{\beamergotobutton{epi. cusco.}} 
			\end{itemize} 
			\item Cobertura de Vacunación ({\color{mycolor5}Nuevo} \hyperlink{cobertura_vacuna}{\beamergotobutton{cobertura}}). 
			\item Indicadores de gestión hospitalaria.
			\begin{itemize}
				\item Ocupación camas UCI y no UCI a nivel regional y por hospital. \hyperlink{camas}{\beamergotobutton{camas hosp.}} 
			\end{itemize}
			\item Indicadores por provincias.
			\begin{itemize}
				\item Incidencia, tasa de mortalidad, tasa de letalidad, tasa de positividad PCR, AG, y exceso de mortalidad. \hyperlink{provincias}{\beamergotobutton{provincias}}
			\end{itemize}
			\item Resumen y recomendaciones. \hyperlink{resumen}{\beamergotobutton{resumen}} \hyperlink{recomendaciones}{\beamergotobutton{recomendaciones}}
			\item Links útiles. \hyperlink{links}{\beamergotobutton{links}} \hfill \hyperlink{vacunas_60}{\beamergotobutton{apendice}}
		\end{itemize}
	\end{frame}
	
	%------------------------------------------------------------------------------------------------------------------------------------------------------------------------------------------------------------------------------------------
	% SECCIÓN 1: Indicadores Epidemiológicos, Nivel Nacional
	%------------------------------------------------------------------------------------------------------------------------------------------------------------------------------------------------------------------------------------------
	\section{Indicadores Epidemiológicos, Nivel Nacional}

	\begin{frame}[label=epi_nacional]
	\frametitle{Tendencia de Variantes en Lima y la Sierra Selva Sur}
	\vspace{-.5cm}
	\begin{figure}
		\centering
		\begin{subfigure}[b]{0.45\textwidth}
			\centering
			\includegraphics[width=\textwidth]{../sala_nacional/variantes_lima.png}
			\caption{Lima}
			%\label{fig:}
		\end{subfigure}
		\hfill
		\begin{subfigure}[b]{0.48\textwidth}
			\centering
			\includegraphics[width=\textwidth]{../sala_nacional/variantes_sierra_sur.png}
			\caption{Sierra Selva Sur}
			%\label{fig:70 a 79 años}
		\end{subfigure}
	\end{figure}
	{\tiny Fuente: Lescano, Orellana, Fano, Pino, y Flores. Situación Epidemiológica de la COVID-19 al 23 de octubre del 2021. \\
		Sierra Selva Sur: Apurimac, Ayacucho, Cusco, Madre de Dios, Puno. \\
		En Lima: Delta domina totalmente, ($ \sim100\% $), pocas muestras. \\
		Delta domina completamente en Lima y todas las macro-regiones. \\} 
	\vspace{0.01cm}
	$\rightarrow$ La variante delta (línea verde) se presenta con mayor frecuencia en la Sierra Selva Sur como en la capital. 
\end{frame}

\begin{frame}
	\frametitle{Cobertura de Vacunación COVID-19 por Regiones}
	\vspace{-.5cm}
	\begin{center}
		\includegraphics[width=0.9\linewidth]{../sala_nacional/vacunas_nacional.png} 
	\end{center}
	\begin{tikzpicture}[overlay]
		\draw[mycolor2, very thick] (4.6,3.17) circle [x radius=4.0cm, y radius=.11cm, rotate=0];
	\end{tikzpicture}
	{\tiny Fuente: Juan Carbajal ($@$juank23\_7). Consultado el 23 de octubre del 2021. \\} 
	\vspace{0.01cm}
	$\rightarrow$ Cusco está en el puesto 14.
\end{frame}

\begin{frame}
		\frametitle{Mortalidad por Regiones}
		\vspace{-.2cm}
		\begin{center}
			\includegraphics[width=0.75\linewidth]{../sala_nacional/mortalidad_regional.png}
		\end{center}
		\begin{tikzpicture}[overlay]
		\draw[mycolor2, ultra thick] (3.5,4.70) circle [x radius=2.cm, y radius=.13cm, rotate=0];
		\end{tikzpicture}
		{\tiny Fuente: Lescano, Orellana, Fano, Pino, y Flores. Situación Epidemiológica de la COVID-19 al 23 de octubre del 2021. \\}
\vspace{0.01cm}
$\rightarrow$ Cusco se mantiene en el puesto 7, con $ 3,233 $ defunciones por COVID-19 por millón de habitantes. \\
	\end{frame}

\begin{frame}
	\frametitle{Defunciones Semanales por Regiones}
\vspace{-.5cm}
\begin{center}
	\includegraphics[width=0.6\linewidth]{../sala_nacional/defunciones_regional.png}
\end{center}
\begin{tikzpicture}[overlay]
	\draw[mycolor2, ultra thick] (5.3,3.62) circle [x radius=3.8cm, y radius=.2cm, rotate=0];
\end{tikzpicture}
{\tiny Fuente: Lescano, Orellana, Fano, Pino, y Flores. Situación Epidemiológica de la COVID-19 al 23 de octubre del 2021.\\
	Nota 1: Records semanales de pandemia (rojo) y 2a ola (amarillo). \\
Nota 2: Sube $ 14.5.4\% $, 21 fallecidos más. \\
Alza alta en Lima metropolitana  y resto del país. 
Mayor alza en costa central. \\
\textbf{Riesgos}: Lima metropolitana con tres subidas en cinco semanas. Mayor disparidad esta semana. Piura sube, nunca se acercó a cero, otros indicadores suben. \\
\textbf{Estancamiento}: Casi todas las alzas son leves. Trece regiones suben poco o están estancadas($\pm 1$).\\} 
\vspace{0.01cm}
$\searrow$ {\color{mycolor5}Cusco}. Las defunciones representan un $3.80\% $ del pico de la segunda ola, la semana anterior fue de $4.43\%$. \\
\end{frame}
	
	\begin{frame}
		\frametitle{Número Reproductivo Efectivo por Regiones}
		\vspace{-.5cm}
		\begin{center}
			\includegraphics[width=0.71\linewidth]{../sala_nacional/rt_regional.png}
		\end{center}
		\begin{tikzpicture}[overlay]
		\draw[mycolor2, ultra thick] (5.75,3.87) circle [x radius=3.4cm, y radius=.18cm, rotate=0];
		\end{tikzpicture}
	{\tiny Fuente: CDC MINSA. Reporte de Vigilancia COVID-19, Perú 2021. Actualización: 2021-10-21.\\}
	\vspace{0.01cm}
	$\rightarrow$ El número reproductivo efectivo (RT) indica la dinámica de la enfermedad; si es menor a $ 1 $, existe control de la epidemia. \\
	$\nearrow$ Cusco sube a la posición 12 con un RT de $1.01$ $ [0.88-1.14]$.

	\end{frame}

	\begin{frame}
	\frametitle{Ocupación de Camas UCI Comparativo por Regiones}
	\vspace{-.5cm}
	\begin{center}
		\includegraphics[width=0.60\linewidth]{../sala_nacional/camas_regional.png}
	\end{center}
	\begin{tikzpicture}[overlay]
	\draw[mycolor2, ultra thick] (4.05,4.0) circle [x radius=.55cm, y radius=.55cm, rotate=0];
	\end{tikzpicture}
	{\tiny Fuente: Lescano, Orellana, Fano, Pino, y Flores. Situación Epidemiológica de la COVID-19 al 23 de octubre del 2021. \\
	Nota 1: Perú: 1039 camas libres, baja $3.3\%$. Lima: 296 camas libres, baja $ 9.5\% $. \\
	Suben $ >1\% $ {\color{red}*}  Huancavelica (4.5\%), Ayacucho (1.8\%), Madre de Dios (3.2\%), Arequipa (1.3\%) y Moquegua (5.2\%) en el sur, y Tumbes (16.7\%), Piura (6.0\%), La Libertad (4.8\%) y Lima región (4.2\%) en la costa norte y centro. Tumbes y Piura repiten {\color{red}**}. Semanas previas: 6, 5, 5 y 7 regiones. \\ 
	Muy > que media: Piura (65\%), Ancash (56\%), Lima metropolitana (55\%), Callao (59\%), Loreto (54\%) y {\color{mycolor5}Cusco} (57\%).\\}
	\vspace{0.01cm}
	$\rightarrow$ Cusco mantiene el nivel de ocupación con respecto a la semana anterior.
\end{frame}

	\begin{frame}
	\frametitle{Ocupación de Camas UCI por Regiones}
	\vspace{-.5cm}
	\begin{center}
		\includegraphics[width=0.75\linewidth]{../sala_nacional/uci_regional.png}
	\end{center}
	\begin{tikzpicture}[overlay]
	\draw[mycolor2, thick] (5.1,2.25) circle [x radius=3.9cm, y radius=.15cm, rotate=0];
\end{tikzpicture}
	{\tiny Fuente: Lescano, Orellana, Fano, Pino, y Flores. Situación Epidemiológica de la COVID-19 al 23 de octubre del 2021.\\}
	\vspace{0.1cm}
	$\searrow$ Cusco pasa a la posición 21.
\end{frame}
	
	\begin{frame}
		\frametitle{Ocupación de Camas No UCI por Regiones}
		\vspace{-.5cm}
		\begin{center}
			\includegraphics[width=0.75\linewidth]{../sala_nacional/nouci_regional.png} 
		\end{center}
		\begin{tikzpicture}[overlay]
		\draw[mycolor2, thick] (3.5,2.89) circle [x radius=2.0cm, y radius=.15cm, rotate=0];
		\end{tikzpicture}
		{\tiny Fuente: Lescano, Orellana, Fano, Pino, y Flores. Situación Epidemiológica de la COVID-19 al 23 de octubre del 2021. \\} 
		\vspace{0.01cm}
		$\nearrow$ Cusco se pasa al puesto 18.
		
		\hyperlink{indice}{\beamergotobutton{Índice}} 
	\end{frame}


	%------------------------------------------------------------------------------------------------------------------------------------------------------------------------------------------------------------------------------------------
% SECCIÓN 2: Indicadores Epidemiológicos, Región Cusco
%------------------------------------------------------------------------------------------------------------------------------------------------------------------------------------------------------------------------------------------
\section{Indicadores Epidemiológicos, Región Cusco}

	\begin{frame}[label=epi_cusco]
		\frametitle{Curva Epidémica, Casos Positivos Semanales}
		\vspace{-.5cm}
		\begin{center}
			\includegraphics[width=.9\linewidth]{../figuras/positivos_20_21.pdf}
		\end{center}
		{\tiny Fuente de datos: SISCOVID, NOTICOVID. \\
			Nota: {\color{mycolor1} --- ---: Navidad}, {\color{mycolor1} - -: Año Nuevo}, {\color{mycolor2} - -: Semana Santa}, {\color{mycolor3} - -: Elecciones Primera Vuelta}, {\color{mycolor4} $- \cdot$: Elecciones Segunda Vuelta}. \\}
	\end{frame}
	
	\begin{frame}
		\frametitle{Curva Epidémica de Sintomáticos y Asintomáticos Semanales}
		\vspace{-.5cm}
		\begin{center}
			\includegraphics[width=0.8\linewidth]{../figuras/sintomaticos_20_21.pdf}
		\end{center} 
		{\tiny Fuente de datos: SISCOVID, NOTICOVID.}
	\end{frame}
	
	\begin{frame}
		\frametitle{Curva Epidémica de Sintomáticos por Tipo de Prueba}
		\vspace{-.5cm}
		\begin{center}
			\includegraphics[width=0.9\linewidth, trim={0cm .5cm 0cm 0.2cm},clip]{../figuras/sinto_prueba_20_21.pdf}
		\end{center}
		{\tiny Fuente de datos: SISCOVID, NOTICOVID.}
	\end{frame}
	
	\begin{frame}
		\frametitle{Tasa de Crecimiento de Casos por Semana}
		\vspace{-.5cm}
		\begin{center}
			\includegraphics[width=0.9\linewidth]{../figuras/positivos_crecimiento_2021.pdf}
		\end{center} 
		{\tiny Fuente de datos: SISCOVID, NOTICOVID. \\
	Nota:{\color{mycolor1} - -: Año Nuevo}, {\color{mycolor2} - -: Semana Santa}, {\color{mycolor3} - -: Elecciones Primera Vuelta}, {\color{mycolor4} $- \cdot$: Elecciones Segunda Vuelta}. \\}
	\end{frame}
	
	\begin{frame}
		\frametitle{Tasa de Positividad Semanal por Tipo de Prueba, 2021}
		\vspace{-.5cm}
		\begin{center}
			\includegraphics[width=0.9\linewidth]{../figuras/positividad_ambas.pdf}
		\end{center}
		{\tiny Fuente de datos: SISCOVID, NOTICOVID.}
	\end{frame}

	\begin{frame}[label=variantes]
	\frametitle{Tendencia de Variantes en la Región Cusco, 2021}
	\vspace{-.5cm}
	\begin{center}
		\includegraphics[width=0.9\linewidth]{../figuras/variantes.pdf}
	\end{center}
	{\tiny Fuente de datos: NETLAB Cusco, UPCH.}
	\end{frame}

	\begin{frame}[label=mapa_variantes]
	\frametitle{Cantidad de Casos Variantes y Dispersión Geográfica en las Provincias de Cusco, 2021}
	\begin{center}
		\includegraphics[width=0.4\linewidth]{../figuras/variantes_provincial.pdf}
	\end{center}
	{\tiny Fuente de datos: NETLAB Cusco, UNSAAC, UPCH.}
	
	Ver detalles para cada la Provincia de Cusco, Distritos de la Región, y por Tipo de variantes haciendo clic en los siguientes enlaces:
	\hyperlink{mapa_provincia_cusco}{\beamergotobutton{Prov. Cusco}} \hyperlink{mapa_distrital}{\beamergotobutton{Distritos de Cusco}} \hyperlink{mapa_lambda}{\beamergotobutton{Lambda}}
	\hyperlink{mapa_gamma}{\beamergotobutton{Gamma}}
	\hyperlink{mapa_delta}{\beamergotobutton{Delta}}
\end{frame}

\begin{frame}
	\frametitle{Propagación a Nivel Distrital}
	\vspace{-.5cm}
	\begin{center}
		\includegraphics[width=0.75\linewidth, trim={0cm .5cm 0cm 0.2cm},clip]{../sala_nacional/rt_cusco.png}
	\end{center}
	{\tiny Fuente: CDC MINSA. Reporte de Vigilancia COVID-19, Perú 2021. Actualización: 2021-10-21. Reporte de número reproductivo efectivo (Rt) de COVID-19. Disponible haciendo clic en el siguiente enlace: \href{https://www.dge.gob.pe/portalnuevo/informacion-publica/reporte-de-numero-reproductivo-efectivo-rt/}{CDC-Rt}. \\}
	\vspace{0.01cm}
	$\rightarrow$ Para ver el número exacto de RT por distrito, haga clic en el siguiente link: \href{https://www.dge.gob.pe/portalnuevo/informacion-publica/reporte-de-numero-reproductivo-efectivo-rt/}{\color{mycolor3}reporte-vigilancia-minsa}. \\
\end{frame}
	
	\begin{frame}
		\frametitle{Defunciones Semanales por COVID-19}
		\vspace{-.5cm}
		\begin{center}
			\includegraphics[width=0.9\linewidth, trim={0cm .5cm 0cm 0.2cm},clip]{../figuras/defunciones_20_21.pdf}
		\end{center}
		{\tiny Fuente de datos: SINADEF.\\
		Nota: {\color{mycolor1} --- ---: Navidad}, {\color{mycolor1} - -: Año Nuevo}, {\color{mycolor2} - -: Semana Santa}, {\color{mycolor3} - -: Elecciones Primera Vuelta}, {\color{mycolor4} $- \cdot$: Elecciones Segunda Vuelta}. \\}
	\end{frame}
	
	\begin{frame}
		\frametitle{Tasa de Crecimiento de Defunciones por Semana}
		\vspace{-.5cm}
		\begin{center}
			\includegraphics[width=0.9\linewidth]{../figuras/defunciones_crecimiento_2021.pdf}
		\end{center} 
		{\tiny Fuente de datos: SINADEF. \\
			Nota: {\color{mycolor1} - -: Año Nuevo}, {\color{mycolor2} - -: Semana Santa}, {\color{mycolor3} - -: Elecciones Primera Vuelta}, {\color{mycolor4} $- \cdot$: Elecciones Segunda Vuelta}. \\}
	\end{frame}

	\begin{frame}
		\frametitle{Exceso de Defunciones por Todas las Causas}
		\vspace{-.5cm}
		\begin{center}
			\includegraphics[width=0.9\linewidth]{../figuras/exceso_region.pdf}
		\end{center}
		{\tiny Fuente de datos: SINADEF.} 
	\end{frame}
	
	\begin{frame}
		\frametitle{Mortalidad por Grupos de Edad}
		\vspace{-.5cm}
		\begin{center}
			\includegraphics[width=0.9\linewidth]{../figuras/mortalidad_edad.pdf}
		\end{center}
		{\tiny Fuente de datos: SINADEF, Dirección Ejecutiva de Atención Integral de Salud} 
	\end{frame}

\begin{frame}
		\frametitle{Mortalidad por Grupos de Edad (e Hitos de Vacunación)}
	\vspace{-.5cm}
\begin{figure}
	\centering
	\begin{subfigure}[b]{0.3\textwidth}
		\centering
		\includegraphics[width=\textwidth]{../figuras/mortalidad_edad_80.pdf}
		\caption{Más de 80 años}
		%\label{fig:}
	\end{subfigure}
	\hfill
	\begin{subfigure}[b]{0.3\textwidth}
		\centering
		\includegraphics[width=\textwidth]{../figuras/mortalidad_edad_70.pdf}
		\caption{70 a 79 años}
		%\label{fig:70 a 79 años}
	\end{subfigure}
	\hfill
	\begin{subfigure}[b]{0.3\textwidth}
		\centering
		\includegraphics[width=\textwidth]{../figuras/mortalidad_edad_60.pdf}
		\caption{60 a 69 años}
		%\label{fig:60 a 69 años}
	\end{subfigure}
\vspace{10mm}
	\begin{subfigure}[b]{0.3\textwidth}
	\centering
	\includegraphics[width=\textwidth]{../figuras/mortalidad_edad_50.pdf}
	\caption{50 a 59 años}
	%\label{fig:50 a 59 años}
\end{subfigure}
	\hfill
\begin{subfigure}[b]{0.3\textwidth}
	\centering
	\includegraphics[width=\textwidth]{../figuras/mortalidad_edad_40.pdf}
	\caption{40 a 49 años}
	%\label{fig:40 a 49 años}
\end{subfigure}
\hfill
\begin{subfigure}[b]{0.3\textwidth}
	\centering
	\includegraphics[width=\textwidth]{../figuras/mortalidad_edad_30.pdf}
	\caption{30 a 39 años}
	%\label{fig:40 a 49 años}
\end{subfigure}
	%\caption{Three simple graphs}
	%\label{fig:three graphs}
\end{figure}
\vspace{-.8cm}
		{\tiny Fuente de datos: SINADEF, Dirección Ejecutiva de Atención Integral de Salud.\\}
		{\tiny Nota: Líneas punteadas (- -) es el inicio de la primera dosis de vacuna contra COVID-19 y línea continua (--) es el inicio de la segunda dosis en el respectivo grupo de edad. Nota 2: Las escalas en el eje y son diferentes.\\}
		$\rightarrow$ {\small La curva de tasa de mortalidad estuvo en descenso y las vacunas contribuyeron a disminuir aún más la misma.} \hyperlink{indice}{\beamergotobutton{Índice}}
\end{frame}

%------------------------------------------------------------------------------------------------------------------------------------------------------------------------------------------------------------------------------------------
% SECCIÓN 2: Indicadores Epidemiológicos, Región Cusco
%------------------------------------------------------------------------------------------------------------------------------------------------------------------------------------------------------------------------------------------

\section{Cobertura de Vacunación, Región Cusco}

\begin{frame}[label=cobertura_vacuna]
	\frametitle{Porcentaje de Cobertura de Vacunación por Grupo de Edad, Región Cusco}
	\vspace{-.5cm}
	\begin{center}
		\includegraphics[width=0.8\linewidth, trim={.2cm .5cm .2cm .2cm},clip]{../figuras/vacunacion_grupo_edad.pdf}
	\end{center}
	{\tiny Fuente de datos: SICOVAC - HIS MINSA, Dirección de Estadística GERESA Cusco.} 
\end{frame}

\begin{frame}[label=cobertura_vacuna_provincias]
	\frametitle{Porcentaje de Cobertura de Vacunación de 80 años a más}
	\vspace{-.5cm}
	\begin{center}
		\includegraphics[width=0.8\linewidth, trim={.2cm .5cm .2cm .2cm},clip]{../figuras/vacunacion_provincial_edad_8.pdf}
	\end{center}
	{\tiny Fuente de datos: SICOVAC - HIS MINSA, Dirección de Estadística GERESA Cusco.} \hyperlink{indice}{\beamergotobutton{Índice}}
	
	Ver detalles de estos indicadores para cada grupo de edad de las provincias haciendo clic en los siguientes enlaces:
	\hyperlink{vacunas_70}{\beamergotobutton{70 a 79 años}}
	\hyperlink{vacunas_60}{\beamergotobutton{60 a 69 años}} \hyperlink{vacunas_50}{\beamergotobutton{50 a 59 añosa}} \hyperlink{vacunas_40}{\beamergotobutton{40 a 49 años}} \hyperlink{vacunas_30}{\beamergotobutton{30 a 39 años}}
	\hyperlink{vacunas_20}{\beamergotobutton{20 a 29 años}} \hyperlink{vacunas_10}{\beamergotobutton{12 a 19 años}}
	
\end{frame}



	%-------------------------------------------------------------------------------------------------------------------------------------------------------------------------------------------------\textit{}-----------------------------------------
	% SECCIÓN 2: Indicadores de Gestión Hospitalaria
	%------------------------------------------------------------------------------------------------------------------------------------------------------------------------------------------------------------------------------------------
	\section{Indicadores de Gestión Hospitalaria}
	
	\begin{frame}[label=camas]
		\frametitle{Total y Ocupación de Camas UCI Hospitalarias, 2021}
		\vspace{-.2cm}
		\begin{center}
			\includegraphics[width=0.9\linewidth, trim={0cm .5cm 0cm 0.2cm},clip]{../figuras/uci.pdf}
			
		\begin{table}[]
			\resizebox{8 cm}{!}{%
				\begin{tabular}{cccc}
					\hline
					\multicolumn{4}{c}{\textbf{UCIN,   SE 42}}                                                   \\ \hline
					TOTAL CAMAS UCIN & CAMAS   UCIN OCUPADAS & CAMAS   UCIN LIBRES & PORCENTAJE   OCUPACIÓN UCIN \\ \hline
					25               & 12                   & 13                  & 61\%                        \\ \hline
				\end{tabular}%
			}
		\end{table}
		\end{center}
		{\tiny Fuente de datos: Referencias, contrareferencias.}
	\end{frame}
	
	\begin{frame}
		\frametitle{Total y Ocupación de Camas Hospitalarias en el Nivel III, 2021}
		\vspace{-.5cm}
		\begin{center}
			\includegraphics[width=0.8\linewidth, trim={0cm .5cm 0cm 0.2cm},clip]{../figuras/nivel_3.pdf}
		\end{center}
		{\tiny Fuente de datos: Referencias, contrareferencias.}
	\end{frame}
	
	\begin{frame}
		\frametitle{Total y Ocupación de Camas Hospitalarias. Hospital Regional, 2021}
		\vspace{-.2cm}
		\begin{center}
			\includegraphics[width=0.9\linewidth, trim={0cm .5cm 0cm 0.2cm},clip]{../figuras/h_regional}
			
			\begin{table}[]
				\resizebox{8 cm}{!}{%
					\begin{tabular}{cccc}
						\hline
						\multicolumn{4}{c}{\textbf{UCIN   HOSPITAL REGIONAL, SE 42}}                                 \\ \hline
						TOTAL CAMAS UCIN & CAMAS   UCIN OCUPADAS & CAMAS   UCIN LIBRES & PORCENTAJE   OCUPACIÓN UCIN \\ \hline
						6               & 5                    & 1                  & 86\%                        \\ \hline
					\end{tabular}%
				}
			\end{table}
			
		\end{center}
		{\tiny Fuente de datos: Referencias, contrareferencias.}
	\end{frame}
	
	\begin{frame}
		\frametitle{Total y Ocupación de Camas Hospitalarias. Hospital Antonio Lorena, 2021}
		\vspace{-.2cm}
		\begin{center}
			\includegraphics[width=0.9\linewidth, trim={0cm .5cm 0cm 0.2cm},clip]{../figuras/h_lorena}
			
			\begin{table}[]
				\resizebox{8 cm}{!}{%
					\begin{tabular}{cccc}
						\hline
						\multicolumn{4}{c}{\textbf{UCIN   HOSPITAL ANTONIO LORENA, SE 42}}                           \\ \hline
						TOTAL CAMAS UCIN & CAMAS   UCIN OCUPADAS & CAMAS   UCIN LIBRES & PORCENTAJE   OCUPACIÓN UCIN \\ \hline
						5                & 1                    & 4                  & 43\%                        \\ \hline
					\end{tabular}%
				}
			\end{table}
			
		\end{center}
		{\tiny Fuente de datos: Referencias, contrareferencias.}
	\end{frame}
	
	\begin{frame}
		\frametitle{Total y Ocupación de Camas Hospitalarias, Hospital Nacional Adolfo Guevara Velasco, 2021}
		\vspace{-.2cm}
		\begin{center}
			\includegraphics[width=0.9\linewidth, trim={0cm .5cm 0cm 0.2cm},clip]{../figuras/h_adolfo}
			
			\begin{table}[]
				\resizebox{8 cm}{!}{%
					\begin{tabular}{cccc}
						\hline
						\multicolumn{4}{c}{\textbf{UCIN   HOSPITAL ADOLFO GUEVARA, SE 42}}                           \\ \hline
						TOTAL CAMAS UCIN & CAMAS   UCIN OCUPADAS & CAMAS   UCIN LIBRES & PORCENTAJE   OCUPACIÓN UCIN \\ \hline
						5               & 1                     & 4                   & 43\%                        \\ \hline
					\end{tabular}%
				}
			\end{table}
			
		\end{center}
		{\tiny Fuente de datos: Referencias, contrareferencias}
	\end{frame}
	
	\begin{frame}
		\frametitle{Total y Ocupación de Camas Hospitalarias en el Nivel II, 2021}
		\vspace{-.5cm}
		\begin{center}
			\includegraphics[width=0.8\linewidth, trim={0cm .5cm 0cm 0.2cm},clip]{../figuras/nivel_2.pdf}
		\end{center}
		{\tiny Fuente de datos: Referencias, contrareferencias.} \hyperlink{indice}{\beamergotobutton{Índice}} 
	\end{frame}
	
	
%-------------------------------------------------------------------------------------------------------------------------------------------------------------------------------------------------\textit{}-----------------------------------------
% SECCIÓN 3: Indicadores por Provincias
%------------------------------------------------------------------------------------------------------------------------------------------------------------------------------------------------------------------------------------------
\section{Indicadores por Provincias}
	\begin{frame}[label=semaforo]
		\frametitle{Tasa de Incidencia por Provincias, 2021}
		\vspace{-.5cm}
		
		% en el input de las tablas sólo debe comenzar y terminar con tabular, borrar el tabular de input de la tabla
		\begin{table}[]
			\resizebox{\textwidth}{!}{%
					\begin{tabular}{lrccclr}
		\rowcolor[HTML]{DCE6F1} 
		\multicolumn{1}{c}{\cellcolor[HTML]{DCE6F1}\textbf{PROVINCIA}} & \multicolumn{1}{c}{\cellcolor[HTML]{DCE6F1}\textbf{Población}} & \textbf{PM+}                                                & \textbf{PA+}         & \textbf{Prueba rápida +} & \multicolumn{1}{c}{\cellcolor[HTML]{DCE6F1}\textbf{Total de casos}} & \multicolumn{1}{c}{\cellcolor[HTML]{DCE6F1}\textbf{Incidencia x 10,000 hab}} \\
		\cellcolor[HTML]{FF5050}CUSCO                                  & 463,656                                                        & 10,638                                                      & 15,402               & 15959                    & 41,999                                                              & 905.82                                                                       \\
		\cellcolor[HTML]{F4B084}LA   CONVENCION                        & 185,793                                                        & 721                                                         & 5,136                & 4808                     & 10,665                                                              & 574.03                                                                       \\
		\cellcolor[HTML]{F4B084}URUBAMBA                               & 66,439                                                         & 149                                                         & 1,563                & 1441                     & 3,153                                                               & 474.57                                                                       \\
		\cellcolor[HTML]{FFE699}CANCHIS                                & 105,049                                                        & 212                                                         & 2,489                & 1624                     & 4,325                                                               & 411.71                                                                       \\
		\cellcolor[HTML]{FFE699}ANTA                                   & 57,731                                                         & 404                                                         & 998                  & 956                      & 2,358                                                               & 408.45                                                                       \\
		\cellcolor[HTML]{FFE699}QUISPICANCHI                           & 92,566                                                         & 476                                                         & 1,554                & 991                      & 3,021                                                               & 326.36                                                                       \\
		\cellcolor[HTML]{FFE699}ESPINAR                                & 71,304                                                         & 15                                                          & 999                  & 1061                     & 2,075                                                               & 291.01                                                                       \\
		\cellcolor[HTML]{FFE699}ACOMAYO                                & 28,477                                                         & 16                                                          & 422                  & 268                      & 706                                                                 & 247.92                                                                       \\
		\cellcolor[HTML]{FFE699}CALCA                                  & 76,462                                                         & 110                                                         & 1,108                & 610                      & 1,828                                                               & 239.07                                                                       \\
		\cellcolor[HTML]{FFE699}CHUMBIVILCAS                           & 84,925                                                         & 66                                                          & 769                  & 1141                     & 1,976                                                               & 232.68                                                                       \\
		\cellcolor[HTML]{C6E0B4}CANAS                                  & 40,420                                                         & 49                                                          & 520                  & 268                      & 837                                                                 & 207.08                                                                       \\
		\cellcolor[HTML]{C6E0B4}PARURO                                 & 31,264                                                         & 53                                                          & 311                  & 203                      & 567                                                                 & 181.36                                                                       \\
		\cellcolor[HTML]{C6E0B4}PAUCARTAMBO                            & 52,989                                                         & 105                                                         & 597                  & 247                      & 949                                                                 & 179.09                                                                       \\
		& \multicolumn{1}{l}{}                                           & \multicolumn{1}{l}{}                                        & \multicolumn{1}{l}{} & \multicolumn{1}{l}{}     &                                                                     & \multicolumn{1}{l}{}                                                         \\
		\rowcolor[HTML]{DDEBF7} 
		\textbf{Total   general}                                       & \textbf{1,357,075}                                             & \multicolumn{1}{r}{\cellcolor[HTML]{DDEBF7}\textbf{13,014}} & \textbf{31,868}      & \textbf{29,577}          & \textbf{74,459}                                                     & \textbf{548.67}                                                             
	\end{tabular}
			}
		\end{table}
		{\tiny PA: Prueba Antigénica, PM: Prueba Molecular}\\[0.5 cm]
		{\tiny Fuente de datos: SISCOVID, NOTICOVID.}
	\end{frame}
	
	\begin{frame}
		\frametitle{Tasa de Positividad por Provincias, 2021}
		\vspace{-.5cm}
		
		% en el input de las tablas sólo debe comenzar y terminar con tabular, borrar el tabular de input de la tabla
		\begin{table}[]
			\resizebox{\textwidth}{!}{%
				\begin{tabular}{llllll}
	\rowcolor[HTML]{DDEBF7} 
	\multicolumn{1}{c}{\cellcolor[HTML]{DDEBF7}\textbf{PROVINCIA}} & \multicolumn{1}{c}{\cellcolor[HTML]{DDEBF7}\textbf{Total de pruebas}} & \multicolumn{1}{c}{\cellcolor[HTML]{DDEBF7}\textbf{Total de PM}} & \multicolumn{1}{c}{\cellcolor[HTML]{DDEBF7}\textbf{Tasa de positividad general}} & \multicolumn{1}{c}{\cellcolor[HTML]{DDEBF7}\textbf{Tasa de positividad PM}} & \multicolumn{1}{c}{\cellcolor[HTML]{DDEBF7}\textbf{Tasa de positividad Pruebas AG}} \\
	\cellcolor[HTML]{FF5050}LA CONVENCIÓN                          & 24,318                                                                & 2,007                                                            & 43.9                                                                             & 35.9                                                                        & 24.5                                                                                \\
	\cellcolor[HTML]{FF5050}ACOMAYO                                & 1,625                                                                 & 94                                                               & 43.4                                                                             & 17.0                                                                        & 23.9                                                                                \\
	\cellcolor[HTML]{FF5050}CALCA                                  & 4,264                                                                 & 421                                                              & 42.9                                                                             & 26.1                                                                        & 23.3                                                                                \\
	\cellcolor[HTML]{FF5050}URUBAMBA                               & 6,975                                                                 & 1,102                                                            & 45.2                                                                             & 13.5                                                                        & 22.8                                                                                \\
	\cellcolor[HTML]{FF5050}ANTA                                   & 5,164                                                                 & 740                                                              & 45.7                                                                             & 54.6                                                                        & 21.3                                                                                \\
	\cellcolor[HTML]{FF5050}QUISPICANCHI                           & 6,619                                                                 & 1,413                                                            & 45.6                                                                             & 33.7                                                                        & 20.7                                                                                \\
	\cellcolor[HTML]{FF5050}PAUCARTAMBO                            & 1,613                                                                 & 295                                                              & 58.8                                                                             & 35.6                                                                        & 19.7                                                                                \\
	\cellcolor[HTML]{FF5050}CANCHIS                                & 8,913                                                                 & 538                                                              & 48.5                                                                             & 39.4                                                                        & 19.7                                                                                \\
	\cellcolor[HTML]{FF5050}CANAS                                  & 2,004                                                                 & 86                                                               & 41.8                                                                             & 57.0                                                                        & 17.3                                                                                \\
	\cellcolor[HTML]{F8CBAD}CUSCO                                  & 115,793                                                               & 30,396                                                           & 36.3                                                                             & 35.0                                                                        & 15.7                                                                                \\
	\cellcolor[HTML]{F8CBAD}PARURO                                 & 1,517                                                                 & 148                                                              & 37.4                                                                             & 35.8                                                                        & 15.1                                                                                \\
	\cellcolor[HTML]{F8CBAD}CHUMBIVILCAS                           & 4,648                                                                 & 172                                                              & 42.5                                                                             & 38.4                                                                        & 14.0                                                                                \\
	\cellcolor[HTML]{F8CBAD}ESPINAR                                & 9,008                                                                 & 46                                                               & 23.0                                                                             & 32.6                                                                        & 11.2                                                                                \\
	&                                                                       &                                                                  &                                                                                  &                                                                             &                                                                                     \\
	\rowcolor[HTML]{DDEBF7} 
	\textbf{Total   general}                                       & \textbf{192,461}                                                      & \textbf{37,458}                                                  & \textbf{38.7}                                                                    & \textbf{34.7}                                                               & \textbf{17.7}                                                                      
\end{tabular}
			}
		\end{table}	
		{\tiny Fuente de datos: SISCOVID, NOTICOVID.}
		
	\end{frame}
	
	\begin{frame}
		\frametitle{Defunciones Cero por Provincias por Semana, 2021}
		\vspace{-.5cm}
		
		% en el input de las tablas sólo debe comenzar y terminar con tabular, borrar el tabular de input de la tabla
		\begin{table}[]
			\resizebox{\textwidth}{!}{%
				\begin{tabular}{lcccccccc}
	\textbf{}     & SE-35                    & SE-36                    & SE-37                    & SE-38                    & SE-39                    & SE-40                    & SE-41                    & SE-42                    \\
	\textbf{}     & 29ago-4sep               & 5sep-11sep               & 12sep-18sep              & 19sep-25sep              & 19sep-25sep              & 26sep-9oct               & 10oct-16oct              & 17oct-23oct              \\
	Acomayo       & \cellcolor[HTML]{FCC46C} & 1                        & \cellcolor[HTML]{FCC46C} & \cellcolor[HTML]{FCC46C} & \cellcolor[HTML]{FCC46C} & \cellcolor[HTML]{FCC46C} & \cellcolor[HTML]{FCC46C} & \cellcolor[HTML]{FCC46C} \\
	Anta          & \cellcolor[HTML]{FCC46C} & 1                        & 1                        & \cellcolor[HTML]{FCC46C} & \cellcolor[HTML]{FCC46C} & 1                        & \cellcolor[HTML]{FCC46C} & \cellcolor[HTML]{FCC46C} \\
	Calca         & \cellcolor[HTML]{FCC46C} & 1                        & \cellcolor[HTML]{FCC46C} & \cellcolor[HTML]{FCC46C} & \cellcolor[HTML]{FCC46C} & \cellcolor[HTML]{FCC46C} & \cellcolor[HTML]{FCC46C} & \cellcolor[HTML]{FCC46C} \\
	Canas         & 1                        & \cellcolor[HTML]{FCC46C} & \cellcolor[HTML]{FCC46C} & 1                        & \cellcolor[HTML]{FCC46C} & 1                        & \cellcolor[HTML]{FCC46C} & 1                        \\
	Canchis       &                          & \cellcolor[HTML]{FCC46C} & \cellcolor[HTML]{FCC46C} & 1                        & 2                        & \cellcolor[HTML]{FCC46C} & \cellcolor[HTML]{FCC46C} & 2                        \\
	Chumbivilcas  & \cellcolor[HTML]{FCC46C} & 1                        & 1                        & \cellcolor[HTML]{FCC46C} & \cellcolor[HTML]{FCC46C} & 1                        & 1                        & \cellcolor[HTML]{FCC46C} \\
	Cusco         & 6                        & 7                        & 3                        & 2                        & 3                        & 1                        & 4                        & 3                        \\
	Espinar       & \cellcolor[HTML]{FCC46C} & \cellcolor[HTML]{FCC46C} & \cellcolor[HTML]{FCC46C} & \cellcolor[HTML]{FCC46C} & 1                        & \cellcolor[HTML]{FCC46C} & \cellcolor[HTML]{FCC46C} & \cellcolor[HTML]{FCC46C} \\
	La Convención &                          & 3                        & 2                        & 3                        & 1                        & 2                        & \cellcolor[HTML]{FCC46C} & \cellcolor[HTML]{FCC46C} \\
	Paruro        & \cellcolor[HTML]{FCC46C} & \cellcolor[HTML]{FCC46C} & 1                        & \cellcolor[HTML]{FCC46C} & \cellcolor[HTML]{FCC46C} & \cellcolor[HTML]{FCC46C} & \cellcolor[HTML]{FCC46C} & \cellcolor[HTML]{FCC46C} \\
	Paucartambo   & \cellcolor[HTML]{FCC46C} & \cellcolor[HTML]{FCC46C} & \cellcolor[HTML]{FCC46C} & \cellcolor[HTML]{FCC46C} & \cellcolor[HTML]{FCC46C} & 1                        & \cellcolor[HTML]{FCC46C} & \cellcolor[HTML]{FCC46C} \\
	Quispicanchi  & \cellcolor[HTML]{FCC46C} & \cellcolor[HTML]{FCC46C} & 3                        & \cellcolor[HTML]{FCC46C} & \cellcolor[HTML]{FCC46C} & \cellcolor[HTML]{FCC46C} & 1                        & \cellcolor[HTML]{FCC46C} \\
	Urubamba      & \cellcolor[HTML]{FCC46C} & 2                        & \cellcolor[HTML]{FCC46C} & \cellcolor[HTML]{FCC46C} & 1                        & \cellcolor[HTML]{FCC46C} & 1                        & \cellcolor[HTML]{FCC46C}
\end{tabular}
			}
		\end{table}	
		{\tiny Fuente de datos: SINADEF. \\}
				
		$\rightarrow$ Notablemente, las provincias de Acomayo, Calca, Espinar, Paruro, Paucartambo han tenido \textbf{\color{mycolor5}una} defunción por COVID-19 en las últimas 8 semanas epidemológicas.
		
		$\rightarrow$ La última semana (SE42) sólo se registraron defunciones en Canas, Canchis, y Cusco.
		
	\end{frame}
	
	
	\begin{frame}[label=indicadores_provinciales]
		\frametitle{Tasa de Letalidad y Mortalidad, 2021}
		\vspace{-.5cm}
		
		% en el input de las tablas sólo debe comenzar y terminar con tabular, borrar el tabular de input de la tabla
		\begin{table}[]
			\resizebox{\textwidth}{!}{%
				\begin{tabular}{lccccc}
	\rowcolor[HTML]{DDEBF7} 
	\multicolumn{1}{c}{\cellcolor[HTML]{DDEBF7}\textbf{Provincias}} & \textbf{Población}   & \textbf{Total de  Pruebas} & \textbf{Defunciones} & \textbf{Tasa de letalidad} & \textbf{Tasa de mortalidad x   100,000 hab} \\
	\cellcolor[HTML]{FF5050}CANCHIS                                 & 105,049              & 4,325                      & 291                  & 6.7\%                      & 277.0                                       \\
	\cellcolor[HTML]{FF5050}CUSCO                                   & 463,656              & 41,999                     & 1,220                & 2.9\%                      & 263.1                                       \\
	\cellcolor[HTML]{FF5050}ANTA                                    & 57,731               & 2,358                      & 148                  & 6.3\%                      & 256.4                                       \\
	\cellcolor[HTML]{FF5050}QUISPICANCHI                            & 92,566               & 3,021                      & 218                  & 7.2\%                      & 235.5                                       \\
	\cellcolor[HTML]{F4B084}URUBAMBA                                & 66,439               & 3,153                      & 136                  & 4.3\%                      & 204.7                                       \\
	\cellcolor[HTML]{F4B084}CANAS                                   & 40,420               & 837                        & 67                   & 8.0\%                      & 165.8                                       \\
	\cellcolor[HTML]{F4B084}PARURO                                  & 31,264               & 567                        & 49                   & 8.6\%                      & 156.7                                       \\
	\cellcolor[HTML]{F4B084}LA CONVENCIÓN                           & 185,793              & 10,665                     & 290                  & 2.7\%                      & 156.1                                       \\
	\cellcolor[HTML]{FFE699}CHUMBIVILCAS                            & 84,925               & 1,976                      & 108                  & 5.5\%                      & 127.2                                       \\
	\cellcolor[HTML]{FFE699}PAUCARTAMBO                             & 52,989               & 949                        & 67                   & 7.1\%                      & 126.4                                       \\
	\cellcolor[HTML]{FFE699}ACOMAYO                                 & 28,477               & 706                        & 33                   & 4.7\%                      & 115.9                                       \\
	\cellcolor[HTML]{FFE699}CALCA                                   & 76,462               & 1,828                      & 74                   & 4.0\%                      & 96.8                                        \\
	\cellcolor[HTML]{FFE699}ESPINAR                                 & 71,304               & 2,075                      & 55                   & 2.7\%                      & 77.1                                        \\
	& \multicolumn{1}{l}{} & \multicolumn{1}{l}{}       & \multicolumn{1}{l}{} & \multicolumn{1}{l}{}       & \multicolumn{1}{l}{}                        \\
	\rowcolor[HTML]{DDEBF7} 
	\textbf{Total general}                                          & \textbf{1,357,075}   & \textbf{74,459}            & \textbf{2,756}       & \textbf{3.7\%}             & \textbf{203.1}                             
\end{tabular}
 
			}
		\end{table}
		{\tiny Fuente de datos: SINADEF. \hyperlink{indice}{\beamergotobutton{Índice}} \\} 
		
		Ver detalles de la tendencia (2020 y 2021) de estos indicadores para cada provincia haciendo clic en los siguientes enlaces:\\ \hyperlink{Acomayo}{\beamergotobutton{Acomayo}} \hyperlink{Anta}{\beamergotobutton{Anta}} \hyperlink{Calca}{\beamergotobutton{Calca}} \hyperlink{Canas}{\beamergotobutton{Canas}} \hyperlink{Chumbivilcas}{\beamergotobutton{Chimbivilcas}}
		\hyperlink{Canchis}{\beamergotobutton{Canchis}} \hyperlink{Cusco}{\beamergotobutton{Cusco}}
		\hyperlink{Espinar}{\beamergotobutton{Espinar}}
		\hyperlink{laconvencion}{\beamergotobutton{La Convencion}}
		\hyperlink{Paruro}{\beamergotobutton{Paruro}} \hyperlink{Paucartambo}{\beamergotobutton{Paucartambo}}
		\hyperlink{Quispicanchi}{\beamergotobutton{Quispicanchi}}
		\hyperlink{Urubamba}{\beamergotobutton{Urubamba}}
	\end{frame}

%-------------------------------------------------------------------------------------------------------------------------------------------------------------------------------------------------\textit{}-----------------------------------------
% SECCIÓN 4: Resúmen y Recomendaciones
%------------------------------------------------------------------------------------------------------------------------------------------------------------------------------------------------------------------------------------------
\section{Resumen }

\begin{frame}[label=resumen]
	\frametitle{Análisis Situacional por COVID-19: Resumen}
	\vspace{-.5cm}
	\begin{itemize}
		\item La Región de Cusco aún se ubica en el \textbf{\color{mycolor4}séptimo lugar} de \textbf{\color{mycolor3}mortalidad} acumulada a nivel nacional, con una índice de propagación de $ 1.01 $.
		\item Continua la \textbf{\color{mycolor4}desaceleración sostenida} en el número de \textbf{\color{mycolor3}casos} y defunciones por COVID-19. 
		\item La \textbf{\color{mycolor3}tasa de positividad} para pruebas antigénicas se encuentra en \textbf{\color{mycolor4}descenso} con un $6\%$, manteniéndose en este rango durante las últimas doce semanas. Mientras que, la \textbf{\color{mycolor3}tasa de positividad} para pruebas moleculares disminuyó a un $11\%$ para la SE42.
		\end{itemize}
\end{frame}

\begin{frame}
	\frametitle{Análisis Situacional por COVID-19: Resumen}
	\vspace{-.5cm}
	\begin{itemize}
	\item La ocupación de camas \textbf{\color{mycolor3}UCI} aún es \textbf{\color{mycolor4} alta}, a pesar de haber disminuido a un $74\% $. 
	\item La ocupación de camas \textbf{\color{mycolor3}no UCI}, se ha mantenido en descenso con un $13\%$. \textbf{\color{mycolor4}} 
	\item La \textbf{\color{mycolor3}cobertura de vacunación regional} para COVID-19, muestra una \textbf{\color{mycolor4}cobertura por encima del 70\% para los grupos etarios mayores a 50 años}, con una brecha más amplia en los grupos etarios menores a 39 años.
	
	\end{itemize}
\end{frame}

\begin{frame}
	\frametitle{Análisis Situacional por COVID-19: Resumen}
	\vspace{-.5cm}
	\begin{itemize}
		\item Semáforo epidemiológico regional
		\begin{itemize}
			\item Tasa de crecimiento semanal de \textbf{\color{mycolor4}casos} en \textbf{\color{mycolor3}verde}.
			\item Tasa de crecimiento semanal de \textbf{\color{mycolor4}defunciones} en \textbf{\color{mycolor3}verde}.
			\item Tasa de \textbf{\color{mycolor4}positividad} de pruebas \textbf{\color{mycolor4}antigénicas} en verde y de prueba moleculares en ámbar.
			\item Disponibilidad de \textbf{\color{mycolor4}camas hospitalarias}: no UCI: en \textbf{\color{mycolor3}verde} el III y nivel II.
			\item Disponibilidad de camas hospitalarias: \textbf{\color{mycolor4}UCI} en \textbf{\color{mycolor3}rojo}.
		\end{itemize} 
		\item Semáforo epidemiológico a nivel provincial
		\begin{itemize}
			\item Todas las provincias presentan desaceleración de casos y defunciones para la SE 42. El exceso regional de mortalidad general es de -2.
	
		\end{itemize}
	\end{itemize}
\end{frame}

\begin{frame}[label=recomendaciones]
	\frametitle{Análisis Situacional por COVID-19: Recomendaciones}
	\vspace{-.5cm}
	\begin{itemize}
			\item Reforzar medidas de control por comandos C19 provinciales y distritales
			\item Campañas de comunicación, traducir mensajes claros, consistentes y constantes
			\item Énfasis en distanciamiento social, mascarilla, lavado de manos, aislamiento temprano si hay síntomas
			\item Asegurar aislamiento efectivo a personas con síntomas por 10 días y cuarentena de contactos cercanos por 14 días
			\item Aislamiento desde la llamada, no depender de la prueba
			\item Dar oxímetro, canasta de alimentos, monitoreo telefónico. 
			\item EPP bien usado – considerar que todos los pacientes son potenciales infectados
			\item Detectar y prevenir brotes nosocomiales
			
		\end{itemize} 

\end{frame}

\section{Links Útiles}

\begin{frame}[label=links]
	\frametitle{Links Útiles}
	\vspace{-.5cm}
	\begin{itemize}
		\item Encuentre {\color{mycolor4} información de la pandemia actualizada diaria a nivel regional, provincial, y distrital} en nuestro {\color{mycolor4}\textbf{Dashboard GERESA}} haciendo clic \href{https://geresacusco.shinyapps.io/DASHBOARD_COVID-19_CUSCO/}{\color{mycolor2}aquí}.
		\item Encuentre información actualizada de los {\color{mycolor4}Mapas de Calor COVID-19} en el haciendo clic \href{http://www.diresacusco.gob.pe/diresa/}{\color{mycolor2}aquí}.
		\item Encuentre información diaria del {\color{mycolor4} Resumen de la Sala Situacional COVID-19} de la Región haciendo clic \href{https://app.powerbi.com/view?r=eyJrIjoiZDdiMzA4YWMtZTZmNC00ZWE2LWFmMmYtODkwZmM1ODhiYTljIiwidCI6IjM2NGE0NmEwLTk0YzctNGZkNi1iYTNjLTlmMmQzMjA5YzFlZiJ9}{\color{mycolor2}aquí}.
		\item Encuentre información resumen cuatro semanas de la situación epidemiológica de COVID-19 en los {\color{mycolor4}Boletines COVID-19} haciendo clic \href{https://sites.google.com/view/geresacusco/boletines-epidemiologicos-covid-19}{\color{mycolor2}aquí}. \hyperlink{indice}{\beamergotobutton{Índice}}
	\end{itemize}

\end{frame}


%------------------------------------------------------------------------------------------------------------------------------------------------------------------------------------------------------------------------------------------
% APÉNDICE
%------------------------------------------------------------------------------------------------------------------------------------------------------------------------------------------------------------------------------------------

%\backupbegin

%\setbeamercovered{invisible}
%\begin{frame}[plain,noframenumbering]
%	\titlepage
%\end{frame}

\appendix
\section{Apéndice}

\subsection{Vacunación por Provincias y Grupo de Edad}

\begin{frame}[label=vacunas_70]
	\frametitle{Porcentaje de Cobertura de Vacunación de 70 a 79 años}
	\vspace{-.5cm}
	\begin{center}
		\includegraphics[width=0.8\linewidth, trim={.2cm .5cm .2cm .2cm},clip]{../figuras/vacunacion_provincial_edad_7.pdf}
	\end{center}
	{\tiny Fuente de datos: SICOVAC - HIS MINSA, Dirección de Estadística GERESA Cusco. \\}
\hyperlink{cobertura_vacuna_provincias}{\beamergotobutton{regresar}}
\end{frame}

\begin{frame}[label=vacunas_60]
	\frametitle{Porcentaje de Cobertura de Vacunación de 60 a 69 años, Provincias de Cusco}
	\vspace{-.5cm}
	\begin{center}
		\includegraphics[width=0.8\linewidth, trim={.2cm .5cm .2cm .2cm},clip]{../figuras/vacunacion_provincial_edad_6.pdf}
	\end{center}
	{\tiny Fuente de datos: SICOVAC - HIS MINSA, Dirección de Estadística GERESA Cusco. \\}
	\hyperlink{cobertura_vacuna_provincias}{\beamergotobutton{regresar}}
\end{frame}

\begin{frame}[label=vacunas_50]
	\frametitle{Porcentaje de Cobertura de Vacunación de 50 a 59 años, Provincias de Cusco}
	\vspace{-.5cm}
	\begin{center}
		\includegraphics[width=0.8\linewidth, trim={.2cm .5cm .2cm .2cm},clip]{../figuras/vacunacion_provincial_edad_5.pdf}
	\end{center}
	{\tiny Fuente de datos: SICOVAC - HIS MINSA, Dirección de Estadística GERESA Cusco. \\}
\hyperlink{cobertura_vacuna_provincias}{\beamergotobutton{regresar}}
\end{frame}

\begin{frame}[label=vacunas_40]
	\frametitle{Porcentaje de Cobertura de Vacunación de 40 a 49 años, Provincias de Cusco}
	\vspace{-.5cm}
	\begin{center}
		\includegraphics[width=0.8\linewidth, trim={.2cm .5cm .2cm .2cm},clip]{../figuras/vacunacion_provincial_edad_4.pdf}
	\end{center}
	{\tiny Fuente de datos: SICOVAC - HIS MINSA, Dirección de Estadística GERESA Cusco. \\}
\hyperlink{cobertura_vacuna_provincias}{\beamergotobutton{regresar}}
\end{frame}

\begin{frame}[label=vacunas_30]
	\frametitle{Porcentaje de Cobertura de Vacunación de 30 a 39 años, Provincias de Cusco}
	\vspace{-.5cm}
	\begin{center}
		\includegraphics[width=0.8\linewidth, trim={.2cm .5cm .2cm .2cm},clip]{../figuras/vacunacion_provincial_edad_3.pdf}
	\end{center}
	{\tiny Fuente de datos: SICOVAC - HIS MINSA, Dirección de Estadística GERESA Cusco. \\}
\hyperlink{cobertura_vacuna_provincias}{\beamergotobutton{regresar}}
\end{frame}

\begin{frame}[label=vacunas_20]
	\frametitle{Porcentaje de Cobertura de Vacunación de 20 a 29 años, Provincias de Cusco}
	\vspace{-.5cm}
	\begin{center}
		\includegraphics[width=0.8\linewidth, trim={.2cm .5cm .2cm .2cm},clip]{../figuras/vacunacion_provincial_edad_2.pdf}
	\end{center}
	{\tiny Fuente de datos: SICOVAC - HIS MINSA, Dirección de Estadística GERESA Cusco. \\}
\hyperlink{cobertura_vacuna_provincias}{\beamergotobutton{regresar}}
\end{frame}

\begin{frame}[label=vacunas_10]
	\frametitle{Porcentaje de Cobertura de Vacunación de 12 a 19 años, Provincias de Cusco}
	\vspace{-.5cm}
	\begin{center}
		\includegraphics[width=0.8\linewidth, trim={.2cm .5cm .2cm .2cm},clip]{../figuras/vacunacion_provincial_edad_1.pdf}
	\end{center}
	{\tiny Fuente de datos: SICOVAC - HIS MINSA, Dirección de Estadística GERESA Cusco. \\}
\hyperlink{cobertura_vacuna_provincias}{\beamergotobutton{regresar}}
\end{frame}

\subsection{Acomayo}

\begin{frame}[label=Acomayo]
	\frametitle{Incidencia y Mortalidad, Provincia Acomayo}
	\vspace{-.5cm}
	\begin{center}
		\includegraphics[width=0.8\linewidth, trim={0cm .5cm 0cm 0.2cm},clip]{../figuras/incidencia_mortalidad_20_21_1.pdf}
	\end{center}
	{\tiny Fuente de datos: SISCOVID, NOTICOVID, SINADEF.}
\end{frame}

\begin{frame}
	\frametitle{Tasa de Positividad, Provincia Acomayo}
	\vspace{-.5cm}
	\begin{center}
		\includegraphics[width=0.8\linewidth, trim={0cm .5cm 0cm 0.2cm},clip]{../figuras/positividad_20_21_1.pdf}
	\end{center}
	{\tiny Fuente de datos: SISCOVID, NOTICOVID.}
\end{frame}

\begin{frame}
	\frametitle{Exceso de Defunciones por Todas las Causas, Provincia Acomayo}
	\vspace{-.5cm}
	\begin{center}
		\includegraphics[width=0.8\linewidth, trim={0cm .5cm 0cm 0.2cm},clip]{../figuras/exceso_1.pdf}
	\end{center}
	{\tiny Fuente de datos: SINADEF.}
	
	\hyperlink{indicadores_provinciales}{\beamergotobutton{regresar}}
\end{frame}

\subsection{Anta}

\begin{frame}[label=Anta]
	\frametitle{Incidencia y Mortalidad, Provincia Anta}
	\vspace{-.5cm}
	\begin{center}
		\includegraphics[width=0.8\linewidth, trim={0cm .5cm 0cm 0.2cm},clip]{../figuras/incidencia_mortalidad_20_21_2.pdf}
	\end{center}
	{\tiny Fuente de datos: SISCOVID, NOTICOVID, SINADEF.}
\end{frame}

\begin{frame}
	\frametitle{Tasa de Positividad, Provincia Anta}
	\vspace{-.5cm}
	\begin{center}
		\includegraphics[width=0.8\linewidth, trim={0cm .5cm 0cm 0.2cm},clip]{../figuras/positividad_20_21_2.pdf}
	\end{center}
	{\tiny Fuente de datos: SISCOVID, NOTICOVID.}
\end{frame}

\begin{frame}
	\frametitle{Exceso de Defunciones por Todas las Causas, Provincia Anta}
	\vspace{-.5cm}
	\begin{center}
		\includegraphics[width=0.8\linewidth, trim={0cm .5cm 0cm 0.2cm},clip]{../figuras/exceso_2.pdf}
	\end{center}
	{\tiny Fuente de datos: SINADEF.}
	
	\hyperlink{indicadores_provinciales}{\beamergotobutton{regresar}}
\end{frame}


\subsection{Calca}

\begin{frame}[label=Calca]
	\frametitle{Incidencia y Mortalidad, Provincia Calca}
	\vspace{-.5cm}
	\begin{center}
		\includegraphics[width=0.8\linewidth, trim={0cm .5cm 0cm 0.2cm},clip]{../figuras/incidencia_mortalidad_20_21_3.pdf}
	\end{center}
	{\tiny Fuente de datos: SISCOVID, NOTICOVID, SINADEF.}
\end{frame}

\begin{frame}
	\frametitle{Tasa de Positividad, Provincia Calca}
	\vspace{-.5cm}
	\begin{center}
		\includegraphics[width=0.8\linewidth, trim={0cm .5cm 0cm 0.2cm},clip]{../figuras/positividad_20_21_3.pdf}
	\end{center}
	{\tiny Fuente de datos: SISCOVID, NOTICOVID.}
\end{frame}

\begin{frame}
	\frametitle{Exceso de Defunciones por Todas las Causas, provincia Calca}
	\vspace{-.5cm}
	\begin{center}
		\includegraphics[width=0.8\linewidth, trim={0cm .5cm 0cm 0.2cm},clip]{../figuras/exceso_3.pdf}
	\end{center}
	{\tiny Fuente de datos: SINADEF.}
	
	\hyperlink{indicadores_provinciales}{\beamergotobutton{regresar}}
\end{frame}

\subsection{Canas}

\begin{frame}[label=Canas]
	\frametitle{Incidencia y Mortalidad, Provincia Canas}
	\vspace{-.5cm}
	\begin{center}
		\includegraphics[width=0.8\linewidth, trim={0cm .5cm 0cm 0.2cm},clip]{../figuras/incidencia_mortalidad_20_21_4.pdf}
	\end{center}
	{\tiny Fuente de datos: SISCOVID, NOTICOVID, SINADEF}
\end{frame}

\begin{frame}
	\frametitle{Tasa de positividad, Provincia Canas}
	\vspace{-.5cm}
	\begin{center}
		\includegraphics[width=0.8\linewidth, trim={0cm .5cm 0cm 0.2cm},clip]{../figuras/positividad_20_21_4.pdf}
	\end{center}
	{\tiny Fuente de datos: SISCOVID, NOTICOVID.}
\end{frame}

\begin{frame}
	\frametitle{Exceso de Defunciones por Todas las Causas, provincia Canas}
	\vspace{-.5cm}
	\begin{center}
		\includegraphics[width=0.8\linewidth, trim={0cm .5cm 0cm 0.2cm},clip]{../figuras/exceso_4.pdf}
	\end{center}
	{\tiny Fuente de datos: SINADEF.}
	
	\hyperlink{indicadores_provinciales}{\beamergotobutton{regresar}}
\end{frame}

\subsection{Canchis}

\begin{frame}[label=Canchis]
	\frametitle{Incidencia y Mortalidad, Provincia Canchis}
	\vspace{-.5cm}
	\begin{center}
		\includegraphics[width=0.8\linewidth, trim={0cm .5cm 0cm 0.2cm},clip]{../figuras/incidencia_mortalidad_20_21_5.pdf}
	\end{center}
	{\tiny Fuente de datos: SISCOVID, NOTICOVID, SINADEF}
\end{frame}

\begin{frame}
	\frametitle{Tasa de Positividad, Provincia Canchis}
	\vspace{-.5cm}
	\begin{center}
		\includegraphics[width=0.8\linewidth, trim={0cm .5cm 0cm 0.2cm},clip]{../figuras/positividad_20_21_5.pdf}
	\end{center}
	{\tiny Fuente de datos: SISCOVID, NOTICOVID.}
\end{frame}

\begin{frame}
	\frametitle{Exceso de Defunciones por Todas las Causas, Provincia Canchis}
	\vspace{-.5cm}
	\begin{center}
		\includegraphics[width=0.8\linewidth, trim={0cm .5cm 0cm 0.2cm},clip]{../figuras/exceso_5.pdf}
	\end{center}
	{\tiny Fuente de datos: SINADEF.}
	
	\hyperlink{indicadores_provinciales}{\beamergotobutton{regresar}}
\end{frame}

\subsection{Chumbivilcas}

\begin{frame}[label=Chumbivilcas]
	\frametitle{Incidencia y Mortalidad, Provincia Chumbivilcas}
	\vspace{-.5cm}
	\begin{center}
		\includegraphics[width=0.8\linewidth, trim={0cm .5cm 0cm 0.2cm},clip]{../figuras/incidencia_mortalidad_20_21_6.pdf}
	\end{center}
	{\tiny Fuente de datos: SISCOVID, NOTICOVID, SINADEF}
\end{frame}

\begin{frame}
	\frametitle{Tasa de Positividad, Provincia Chumbivilcas}
	\vspace{-.5cm}
	\begin{center}
		\includegraphics[width=0.8\linewidth, trim={0cm .5cm 0cm 0.2cm},clip]{../figuras/positividad_20_21_6.pdf}
	\end{center}
	{\tiny Fuente de datos: SISCOVID, NOTICOVID.}
\end{frame}

\begin{frame}
	\frametitle{Exceso de Defunciones por Todas las Causas, Provincia Chumbivilcas}
	\vspace{-.5cm}
	\begin{center}
		\includegraphics[width=0.8\linewidth, trim={0cm .5cm 0cm 0.2cm},clip]{../figuras/exceso_6.pdf}
	\end{center}
	{\tiny Fuente de datos: SINADEF.}
	
	\hyperlink{indicadores_provinciales}{\beamergotobutton{regresar}}
\end{frame}

\subsection{Cusco}

\begin{frame}[label=Cusco]
	\frametitle{Incidencia y Mortalidad, Provincia Cusco}
	\vspace{-.5cm}
	\begin{center}
		\includegraphics[width=0.8\linewidth, trim={0cm .5cm 0cm 0.2cm},clip]{../figuras/incidencia_mortalidad_20_21_7.pdf}
	\end{center}
	{\tiny Fuente de datos: SISCOVID, NOTICOVID, SINADEF.}
\end{frame}

\begin{frame}
	\frametitle{Tasa de Positividad, Provincia Cusco}
	\vspace{-.5cm}
	\begin{center}
		\includegraphics[width=0.8\linewidth, trim={0cm .5cm 0cm 0.2cm},clip]{../figuras/positividad_20_21_7.pdf}
	\end{center}
	{\tiny Fuente de datos: SISCOVID, NOTICOVID.}
\end{frame}

\begin{frame}
	\frametitle{Exceso de Defunciones por Todas las Causas, Provincia Cusco}
	\vspace{-.5cm}
	\begin{center}
		\includegraphics[width=0.8\linewidth, trim={0cm .5cm 0cm 0.2cm},clip]{../figuras/exceso_7.pdf}
	\end{center}
	{\tiny Fuente de datos: SINADEF.}
	
	\hyperlink{indicadores_provinciales}{\beamergotobutton{regresar}}
\end{frame}

\subsection{Espinar}

\begin{frame}[label=Espinar]
	\frametitle{Incidencia y Mortalidad, Provincia Espinar}
	\vspace{-.5cm}
	\begin{center}
		\includegraphics[width=0.8\linewidth, trim={0cm .5cm 0cm 0.2cm},clip]{../figuras/incidencia_mortalidad_20_21_8.pdf}
	\end{center}
	{\tiny Fuente de datos: SISCOVID, NOTICOVID, SINADEF.}
\end{frame}

\begin{frame}
	\frametitle{Tasa de Positividad, Provincia Espinar}
	\vspace{-.5cm}
	\begin{center}
		\includegraphics[width=0.8\linewidth, trim={0cm .5cm 0cm 0.2cm},clip]{../figuras/positividad_20_21_8.pdf}
	\end{center}
	{\tiny Fuente de datos: SISCOVID, NOTICOVID.}
\end{frame}

\begin{frame}
	\frametitle{Exceso de Defunciones por Todas las Causas, Provincia Espinar}
	\vspace{-.5cm}
	\begin{center}
		\includegraphics[width=0.8\linewidth, trim={0cm .5cm 0cm 0.2cm},clip]{../figuras/exceso_8.pdf}
	\end{center}
	{\tiny Fuente de datos: SINADEF.}
	
	\hyperlink{indicadores_provinciales}{\beamergotobutton{regresar}}
\end{frame}


\subsection{La Convención}

\begin{frame}[label=laconvencion]
	\frametitle{Incidencia y Mortalidad, Provincia La Convención}
	\vspace{-.5cm}
	\begin{center}
		\includegraphics[width=0.8\linewidth, trim={0cm .5cm 0cm 0.2cm},clip]{../figuras/incidencia_mortalidad_20_21_9.pdf}
	\end{center}
	{\tiny Fuente de datos: SISCOVID, NOTICOVID, SINADEF.}
\end{frame}

\begin{frame}
	\frametitle{Tasa de Positividad,Provincia La Convención}
	\vspace{-.5cm}
	\begin{center}
		\includegraphics[width=0.8\linewidth, trim={0cm .5cm 0cm 0.2cm},clip]{../figuras/positividad_20_21_9.pdf}
	\end{center}
	{\tiny Fuente de datos: SISCOVID, NOTICOVID.}
\end{frame}

\begin{frame}
	\frametitle{Exceso de Defunciones por Todas las Causas, Provincia La Convención}
	\vspace{-.5cm}
	\begin{center}
		\includegraphics[width=0.8\linewidth, trim={0cm .5cm 0cm 0.2cm},clip]{../figuras/exceso_9.pdf}
	\end{center}
	{\tiny Fuente de datos: SINADEF.}
	
	\hyperlink{indicadores_provinciales}{\beamergotobutton{regresar}}
\end{frame}

\subsection{Paruro}

\begin{frame}[label=Paruro]
	\frametitle{Incidencia y Mortalidad, Provincia Paruro}
	\vspace{-.5cm}
	\begin{center}
		\includegraphics[width=0.8\linewidth, trim={0cm .5cm 0cm 0.2cm},clip]{../figuras/incidencia_mortalidad_20_21_10.pdf}
	\end{center}
	{\tiny Fuente de datos: SISCOVID, NOTICOVID, SINADEF.}
\end{frame}

\begin{frame}
	\frametitle{Tasa de Positividad, Provincia Paruro}
	\vspace{-.5cm}
	\begin{center}
		\includegraphics[width=0.8\linewidth, trim={0cm .5cm 0cm 0.2cm},clip]{../figuras/positividad_20_21_10.pdf}
	\end{center}
	{\tiny Fuente de datos: SISCOVID, NOTICOVID.}
\end{frame}

\begin{frame}
	\frametitle{Exceso de Defunciones por Todas las Causas, Provincia Paruro}
	\vspace{-.5cm}
	\begin{center}
		\includegraphics[width=0.8\linewidth, trim={0cm .5cm 0cm 0.2cm},clip]{../figuras/exceso_10.pdf}
	\end{center}
	{\tiny Fuente de datos: SINADEF.}
	
	\hyperlink{indicadores_provinciales}{\beamergotobutton{regresar}}
\end{frame}

\subsection{Paucartambo}

\begin{frame}[label=Paucartambo]
	\frametitle{Incidencia y Mortalidad, Provincia Paucartambo}
	\vspace{-.5cm}
	\begin{center}
		\includegraphics[width=0.8\linewidth, trim={0cm .5cm 0cm 0.2cm},clip]{../figuras/incidencia_mortalidad_20_21_11.pdf}
	\end{center}
	{\tiny Fuente de datos: SISCOVID, NOTICOVID, SINADEF.}
\end{frame}

\begin{frame}
	\frametitle{Tasa de Positividad, Provincia Paucartambo}
	\vspace{-.5cm}
	\begin{center}
		\includegraphics[width=0.8\linewidth, trim={0cm .5cm 0cm 0.2cm},clip]{../figuras/positividad_20_21_11.pdf}
	\end{center}
	{\tiny Fuente de datos: SISCOVID, NOTICOVID.}
\end{frame}

\begin{frame}
	\frametitle{Exceso de Defunciones por Todas las causas, provincia Paucartambo}
	\vspace{-.5cm}
	\begin{center}
		\includegraphics[width=0.8\linewidth, trim={0cm .5cm 0cm 0.2cm},clip]{../figuras/exceso_11.pdf}
	\end{center}
	{\tiny Fuente de datos: SINADEF.}
	
	\hyperlink{indicadores_provinciales}{\beamergotobutton{regresar}}
\end{frame}


\subsection{Quispicanchi}

\begin{frame}[label=Quispicanchi]
	\frametitle{Incidencia y Mortalidad, Provincia Quispicanchi}
	\vspace{-.5cm}
	\begin{center}
		\includegraphics[width=0.8\linewidth, trim={0cm .5cm 0cm 0.2cm},clip]{../figuras/incidencia_mortalidad_20_21_12.pdf}
	\end{center}
	{\tiny Fuente de datos: SISCOVID, NOTICOVID, SINADEF.}
\end{frame}

\begin{frame}
	\frametitle{Tasa de Positividad, Provincia Quispicanchi}
	\vspace{-.5cm}
	\begin{center}
		\includegraphics[width=0.8\linewidth, trim={0cm .5cm 0cm 0.2cm},clip]{../figuras/positividad_20_21_12.pdf}
	\end{center}
	{\tiny Fuente de datos: SISCOVID, NOTICOVID.}
\end{frame}

\begin{frame}
	\frametitle{Exceso de Defunciones por Todas las Causas, Provincia Quispicanchi}
	\vspace{-.5cm}
	\begin{center}
		\includegraphics[width=0.8\linewidth, trim={0cm .5cm 0cm 0.2cm},clip]{../figuras/exceso_12.pdf}
	\end{center}
	{\tiny Fuente de datos: SINADEF.}
	
	\hyperlink{indicadores_provinciales}{\beamergotobutton{regresar}}
\end{frame}

\subsection{Urubamba}

\begin{frame}[label=Urubamba]
	\frametitle{\large Incidencia y Mortalidad, Provincia Urubamba}
	\vspace{-.5cm}
	\begin{center}
		\includegraphics[width=0.8\linewidth, trim={0cm .5cm 0cm 0.2cm},clip]{../figuras/incidencia_mortalidad_20_21_13.pdf}
	\end{center}
	{\tiny Fuente de datos: SISCOVID, NOTICOVID, SINADEF.}
\end{frame}

\begin{frame}
	\frametitle{Tasa de Positividad, Provincia Urubamba}
	\vspace{-.5cm}
	\begin{center}
		\includegraphics[width=0.8\linewidth, trim={0cm .5cm 0cm 0.2cm},clip]{../figuras/positividad_20_21_13.pdf}
	\end{center}
	{\tiny Fuente de datos: SISCOVID, NOTICOVID.}
\end{frame}

\begin{frame}
	\frametitle{Exceso de Defunciones por Todas las Causas, Provincia Urubamba}
	\vspace{-.5cm}
	\begin{center}
		\includegraphics[width=0.8\linewidth, trim={0cm .5cm 0cm 0.2cm},clip]{../figuras/exceso_13.pdf}
	\end{center}
	{\tiny Fuente de datos: SINADEF.} \hyperlink{indice}{\beamergotobutton{Índice}} 
	
	\hyperlink{indicadores_provinciales}{\beamergotobutton{regresar}}
\end{frame}

\subsection{Mapas Variantes}
	\begin{frame}[label=mapa_provincia_cusco]
	\frametitle{Cantidad de Casos Variantes en la Provincia Cusco, 2021}
	\begin{center}
		\includegraphics[width=0.65\linewidth]{../figuras/variantes_distrital_cusco.pdf}
	\end{center}
	{\tiny Fuente de datos: NETLAB Cusco, UNSAAC, UPCH.}
	
	\hyperlink{mapa_variantes}{\beamergotobutton{regresar}}
	\end{frame}

	\begin{frame}[label=mapa_distrital]
	\frametitle{Cantidad de Casos Variantes en los Distritos de la Región Cusco, 2021}
	\begin{center}
		\includegraphics[width=0.55\linewidth]{../figuras/variantes_distrital.pdf}
	\end{center}
	{\tiny Fuente de datos: NETLAB Cusco, UNSAAC, UPCH.}
	
	\hyperlink{mapa_variantes}{\beamergotobutton{regresar}}
	\end{frame}

	\begin{frame}[label=mapa_lambda]
		\frametitle{Cantidad de Casos Variantes \textbf{Lambda} por Provincias, 2021}
		\begin{center}
			\includegraphics[width=0.55\linewidth]{../figuras/variantes_provincial_lambda.pdf}
		\end{center}
		{\tiny Fuente de datos: NETLAB Cusco, UNSAAC, UPCH.}
		
		\hyperlink{mapa_variantes}{\beamergotobutton{regresar}}
	\end{frame}

	\begin{frame}[label=mapa_gamma]
		\frametitle{Cantidad de Casos  Variante \textbf{Gamma} por Provincias, 2021}
		\begin{center}
			\includegraphics[width=0.55\linewidth]{../figuras/variantes_provincial_gamma.pdf}
		\end{center}
		{\tiny Fuente de datos: NETLAB Cusco, UNSAAC, UPCH.}
		
		\hyperlink{mapa_variantes}{\beamergotobutton{regresar}}
	\end{frame}
	
	\begin{frame}[label=mapa_delta]
		\frametitle{Cantidad de Casos  Variante \textbf{Delta} por Provincias, 2021}
		\begin{center}
			\includegraphics[width=0.55\linewidth]{../figuras/variantes_provincial_delta.pdf}
		\end{center}
		{\tiny Fuente de datos: NETLAB Cusco, UNSAAC, UPCH.}
		
		\hyperlink{mapa_variantes}{\beamergotobutton{regresar}}
	\end{frame}
%\backupend

\end{document}
